\documentclass[12pt,a4paper]{article}


\usepackage[utf8]{inputenc}
\usepackage[ngerman]{babel}
\usepackage{amsmath,amssymb}
\usepackage[arrow, matrix, curve]{xy}
\usepackage{amsthm}
\usepackage[top=3cm,left=4cm,right=2cm,bottom=3.5cm]{geometry}
\usepackage{todonotes}
\usepackage{graphicx}
\usepackage[colorlinks=true,linkcolor=blue,citecolor=blue]{hyperref}
\usepackage{float}
\usepackage{dsfont}
\usepackage{subfigure}
\usepackage{fancyhdr}
\usepackage{a4wide}
\usepackage{braket}
\usepackage{cite}
\usepackage{tikz}
\usepackage{verbatim}
\usepackage[noend]{algpseudocode}
\usepackage{algorithm}
\usepackage{authblk}
\renewcommand\Affilfont{\small}
\DeclareMathOperator{\diam}{diam}



\begin{document}
\begin{titlepage}
\begin{center}
\includegraphics[height=.13\textheight]{Logo_uni} \\[5pt]
{\large\text{Fachbereich Mathematik \& Informatik}}  \\[5pt]
{\large\text{Dein Name}}   \\[10pt]
{\large Wintersemester XXXX} \\[1cm]
%%%%%%%%%%%%%%%%%%%%%%%%%%%%%%%%%
{\large{Master-Arbeit zum Thema}} \\[10pt]
	{\Large\textbf{Vorlage für eine Abschlussarbeit}}\\[1cm]
\begin{tabular}[l]{ll}
{\large Betreuer:} & {\large Prof. Dr. C. Komusiewicz}\\
{\large Zweitgutachter:} & {\large XXXXXXXXXXXXXX}
\end{tabular}
\end{center}

	
\vfill
\begin{tabular}[l]{ll}
Name:           & XXXXX\\
Matrikelnummer: & XXXXX\\
Studiengang:    & XXXXX\\
Email: & XXXXX\\
Datum der Abgabe: & XXXXX \\
\end{tabular}
\end{titlepage}

\section*{Selbstständigkeitserklärung}
Hier versicherst du, dass du die Arbeit selbst geschrieben hast.

\section*{Zusammenfassung}
Hier steht auf etwa einer halben Seite eine grobe Beschreibung des Themas sowie eine kurze Zusammenfassung der wichtigsten Ergebnisse. 
Falls die Abschlussarbeit in Englisch geschrieben wird, so muss sowohl eine Zusammenfassung als auch ein Abstract geschrieben werden.

\section*{Danksagung}
Eine Danksagung muss nicht geschrieben werden, kann aber gerne angefertigt werden.





\newgeometry{
  left=4cm,
  right=2cm,
  top=3cm,
  bottom=3.5cm,
  bindingoffset=5mm
}
\begin{verbatim}
\newgeometry{
  left=4cm,
  right=2cm,
  top=3cm,
  bottom=3.5cm,
  bindingoffset=5mm
}
\end{verbatim}
Mit diesem Befehl können die Seitenränder angepasst werden.


\pagenumbering{Roman}
\pagestyle{fancy}
\fancyhf{}
\fancyhead{} 
\fancyhead[R]{\thepage}

\begin{verbatim}
\pagenumbering{Roman}
\pagestyle{fancy}
\fancyhf{}
\fancyhead{} 
\fancyhead[R]{\thepage}
\end{verbatim}
Mit diesem Befehl werden Seiten nun römisch nummeriert und es gibt eine Kopfzeile.


\newpage
\pagestyle{empty}
\tableofcontents
\newpage
\pagenumbering{arabic}    
\pagestyle{fancy}
\makeatletter
\renewcommand{\sectionmark}[1]{\markright{\thesection .~~#1}}
\makeatother

\begin{verbatim}
\newpage
\pagestyle{empty}
\tableofcontents
\newpage
\pagenumbering{arabic}    
\pagestyle{fancy}
\makeatletter
\renewcommand{\sectionmark}[1]{\markright{\thesection .~~#1}}
\makeatother
\end{verbatim}

Mit diesen Befehlen wird ein Inhaltsverzeichnis erstellt. 
Dabei werden alle Seiten bis zum Inhaltsverzeichnis mit römischen Zahlen nummeriert und alle Seiten danach mit arabischen.



\fancyhead[L]{\itshape\nouppercase  \rightmark}
\fancyhead[R]{\thepage}
\begin{verbatim}
\fancyhead[L]{\itshape\nouppercase  \rightmark}
\fancyhead[R]{\thepage}
\end{verbatim}
Hiermit erscheinen die Seitennummern oben rechts.


\section{Einleitung}
In dieser Section motivierst du deine Aufgabenstellung und beschreibst die relevanten Problemdefinitionen.

\subsection{Bekannte Ergebnisse}
Hier gibst du einen Überblick über relevante Ergebnisse der von dir untersuchten Probleme oder anderen verwandeten Problemen.

\subsection{Eigene Ergebnisse}
Hier gibst du eine Zusammenfassung deiner eigenen Ergebnisse.

\subsection{Gliederung}
Hier kannst du nochmal grob in jeweils einem Satz sagen in welcher Section du welche Ergebnisse beweist bzw. auswertest.


\section{Grundlagen}
In dieser Section fasst du alle relevanten Definitionen, Notationen und Problemdefinitionen zusammen.
Dabei kann es sich anbieten jeweils eine Subsection für einzelne Gebiete zu machen. Beispiele sind:
\begin{itemize}
\item Graphnotaion
\item Spezielle Graphen
\item Eigenschaften von Graphen
\item Komplexitätstheorie
\item Parametrisierte Komplexität
\end{itemize}


\section{Hauptteil}
In den folgenden Sections beweist du zunächst deine theoretischen Resultate. 
Diese sollten passend in mehrere Sections aufgeteilt werden.
Falls deine Abschlussarbeit Experimente umfasst, beschreibst du anschließend deine Implementierung und eventuelle Unterschiede der Implementerung zur Theorie.
Anschließend wertest du deine Experimente aus.

\subsection{Allgemeine Hinweise}
  
\begin{itemize}
\item Wir empfehlen Latex zur Anfertigung des Manuskripts.
\item Nutzen Sie gerade zu Beginn der Arbeit Ihre Manuskriptdatei als Notizzettel. Bauen Sie diesen dann Schritt für Schritt in ein fertiges Manuskript um.
\item Es empfiehlt sich eine Itemize-Liste mit den bereits erzielten Ergebnissen und eine mit den Ideen für die noch zu untersuchenden Fragestellungen oder zu erledigenden Aufgaben im Manuskript anzulegen.
\item Schreiben Sie Introduction, Conclusion und Abstract erst zum Schluss, aber sammeln
  Sie durchgängig wichtige Fakten und Beobachtungen in den jeweils betroffenen
  Abschnitten. Wenn Sie zum Beispiel eine interessante offene Frage haben, die Sie im
  Laufe der Arbeit auf keinen Fall mehr bearbeiten wollen, dann notieren Sie diese in der Conclusion.
\item Führen Sie den Leser durch Ihre Arbeit! Erklären Sie etwa zu Beginn einer Section,
  was Sie hier zeigen und welche Struktur die Section hat.
\end{itemize}
\todo[inline]{Referenzen auf gute Latexwebseiten. Empfehlung für Pakete etc.} 

\subsection{Latex, Mathematischer Text und Beweise}

\begin{itemize}
\item Zum Nachschlagen von Latexbefehlen für konkrete Symbole empfehlen wir \url{http://detexify.kirelabs.org}.
\item Zur Bezeichnung von Berechnungsproblemen empfiehlt es sich, einen eigenen Schriftstil zu verwenden, etwa \texttt{\textbackslash{}textsc{}}, dann wäre die Bezeichnung also immer "`\textsc{Approximate Exact Cover}"' anstelle von "`Approximate Exact Cover"'.
Außerdem sollten neu eingeführte Begriffe stets mit \texttt{\textbackslash{}emph{}} im Text hervorgehoben werden.
\item Sätze sollten nie mit mathematischen Symbolen beginnen.
\item Zeilenumbrüche direkt vor oder nach mathematischen Symbolen sollten wenn möglich
  vermieden werden. Etwa sollte bei "`vertex~$v$"' nicht zwischen "`vertex"' und "`$v$"'
  umgebrochen werden. In Latex verhindert man einen Umbruch mit "`$\sim$"'.
\item Zur Notation eines mathematischen Operators, etwa $\diam(G)$ für den Durchmesser eines Graphen, kann man in Latex  \texttt{\textbackslash{}DeclareMathOperator\{\textbackslash{}diam\}\{diam\}} verwenden.  
\item Bei längeren Beweisen sollten Sie die Beweisidee kurz informell erklären, bevor Sie
  den Beweis detailliert beschreiben.
\item   Mittels der folgenden Befehle können Sie Umgebungen für Theoreme, Lemmas, Definitionen und Reduktionsregeln erstellen.
\begin{verbatim}
\newtheorem{theorem}{Theorem}[section]
\newtheorem{lemma}[theorem]{Lemma} 
\theoremstyle{definition}
\newtheorem{defi}[theorem]{Definition}
\newtheorem{reduc}{Reduction Rule}\numberwithin{reduc}{section} 
\end{verbatim} 
\item Für Pseudocode eignen sich folgende Pakete:
\begin{verbatim}
\usepackage[noend]{algpseudocode}
\usepackage{algorithm}
\end{verbatim}

Sie können auch eigene Befehle dem Pseudocode hinzufügen.
\begin{verbatim}
\algnewcommand\AND{\textbf{and}\space}
\algnewcommand\algorithmicforeach{\textbf{for each}}
\algdef{S}[FOR]{ForEach}[1]{\algorithmicforeach\ #1\ \algorithmicdo}
\algrenewcommand\algorithmicprocedure{\textbf{function}}
\end{verbatim}
\algnewcommand\AND{\textbf{and}\space}
\algnewcommand\algorithmicforeach{\textbf{for each}}
\algdef{S}[FOR]{ForEach}[1]{\algorithmicforeach\ #1\ \algorithmicdo}
\algrenewcommand\algorithmicprocedure{\textbf{function}}

Ein Beispiel ist folgender Code:
\begin{verbatim}
\begin{algorithm}[t]
  \caption{An algorithm for finding a dominating clique~$S$. Vertex~$v_i$ is the first vertex of the dominating clique~$S$ in the fixed closure ordering~$\sigma$ of~$G$. Initially we have~$T:=\{v_i\}$.}
  \label{algo-branching-dc}
  \begin{algorithmic}[1]  
  \Procedure{\textit{SolveDC}}{$G, k, T$} 
  \State \algorithmicif\ $k=0$ \AND~$V(G)\neq N[T]$ \algorithmicthen\ \Return No \label{line-cdc-return-no}
  \State \algorithmicif\ $V(G) = N[T]$ \algorithmicthen\ \Return Yes\label{line-cdc-yes}
  \State Compute a vertex~$w$ such that~$v_iw\notin E(G)$ \label{line-cdc-compute-v-j}
  \ForEach{$u\in \bigcap_{x\in T}N(x)\cap N(w)\cap V(G_i)$} \label{line-cdc-branch1} \Comment {$G_i:=G[v_i, \ldots , v_n]$}
     \State \algorithmicif\ \textit{SolveDC}$(G - (N(u)\setminus N[v_i]), k - 1, T\cup\{u\})$ returns Yes \algorithmicthen\ \Return Yes \label{line-cdc-branch2}
  \EndFor
  \State \Return No
  \EndProcedure
  \end{algorithmic}
\end{algorithm}
\end{verbatim}
\begin{algorithm}[t]
  \caption{An algorithm for finding a dominating clique~$S$. Vertex~$v_i$ is the first vertex of the dominating clique~$S$ in the fixed closure ordering~$\sigma$ of~$G$. Initially we have~$T:=\{v_i\}$.}
  \label{algo-branching-dc}
  \begin{algorithmic}[1]  
  \Procedure{\textit{SolveDC}}{$G, k, T$} 
  \State \algorithmicif\ $k=0$ \AND~$V(G)\neq N[T]$ \algorithmicthen\ \Return No \label{line-cdc-return-no}
  \State \algorithmicif\ $V(G) = N[T]$ \algorithmicthen\ \Return Yes\label{line-cdc-yes}
  \State Compute a vertex~$w$ such that~$v_iw\notin E(G)$ \label{line-cdc-compute-v-j}
  \ForEach{$u\in \bigcap_{x\in T}N(x)\cap N(w)\cap V(G_i)$} \label{line-cdc-branch1} \Comment {$G_i:=G[v_i, \ldots , v_n]$}
     \State \algorithmicif\ \textit{SolveDC}$(G - (N(u)\setminus N[v_i]), k - 1, T\cup\{u\})$ returns Yes \algorithmicthen\ \Return Yes \label{line-cdc-branch2}
  \EndFor
  \State \Return No
  \EndProcedure
  \end{algorithmic}
\end{algorithm}
  
  
  
\item Für Abbildungen, Pseudocode und Tabellen gilt die Daumenregel, dass sie am besten am oberen Seitenrand aufgehoben sind, da sie sonst den Textfluss stören. Dies wird mittels~$[t]$ erreicht (siehe dazu den Code der Figure). Zur Erstellung von Abbildungen kann man TikZ verwenden \url{http://mirrors.ctan.org/graphics/pgf/base/doc/pgfmanual.pdf}. Das Bild in Abbildung~\ref{fig:example} wird durch folgenden Code erzeugt.
  \begin{quote}\small
\begin{verbatim}
\begin{figure}[t]
  \centering
  \begin{tikzpicture}[xscale=1.4,yscale=1.5]
    
    \tikzstyle{knoten}=[circle,fill=white,draw=black,thick,
    minimum size=8pt,inner sep=0pt]
    
    \node[knoten,label=left:{$s$}] (a) at (0,1) {};
    \node[knoten,label=left:{}] (b) at (0,0) {};
    \node[knoten,label=below:{}] (c) at (1.3,0) {};
    \node[knoten,label=below:{$t_1$}] (d) at (-1.3,-1) {};
    \node[knoten,label=below:{$t_2$}] (e) at (0,-1)	 {};
    \node[knoten,label=below:{$t_3$}] (f) at (1.3,-1)	 {};	
    \draw[->, line width=2.5pt] (a) edge  node[right] {$6$} (b);
    \draw[->, line width=1pt,bend left] (a) edge  node[above] {$2$} (c);
    \draw[->, line width=1pt, bend right] (a) edge  node[left] {$6$} (d);
    \draw[->, line width=2.5pt] (b) edge  node[above] {$1$} (d);
    \draw[->, line width=2.5pt] (b) edge  node[left] {$2$} (e);
    \draw[->, line width=1pt] (c) edge  node[right] {$2$} (f);
   \end{tikzpicture}
   \caption{Eine mit TikZ erzeugte Abbildung.}
   \label{fig:example}
\end{figure}
\end{verbatim}
  \end{quote}
\end{itemize}
\begin{figure}[t]
  \centering
  \begin{tikzpicture}[xscale=1.4,yscale=1.5]
    
    \tikzstyle{knoten}=[circle,fill=white,draw=black,thick,minimum size=8pt,inner sep=0pt]
    
    \node[knoten,label=left:{$s$}] (a) at (0,1) {};
    \node[knoten,label=left:{}] (b) at (0,0) {};
    \node[knoten,label=below:{}] (c) at (1.3,0) {};
    \node[knoten,label=below:{$t_1$}] (d) at (-1.3,-1) {};
    \node[knoten,label=below:{$t_2$}] (e) at (0,-1)	 {};
    \node[knoten,label=below:{$t_3$}] (f) at (1.3,-1)	 {};	
    \draw[->, line width=2.5pt] (a) edge  node[right] {$6$} (b);
    \draw[->, line width=1pt,bend left] (a) edge  node[above] {$2$} (c);
    \draw[->, line width=1pt, bend right] (a) edge  node[left] {$6$} (d);
    \draw[->, line width=2.5pt] (b) edge  node[above] {$1$} (d);
    \draw[->, line width=2.5pt] (b) edge  node[left] {$2$} (e);
    \draw[->, line width=1pt] (c) edge  node[right] {$2$} (f);
    \end{tikzpicture}
    
    \caption{Eine mit TikZ erzeugte Abbildung.}
    \label{fig:example}
  \end{figure}

\subsection{Literaturangaben und Literaturrecherche}

\begin{itemize}
\item In Latex sollte Bibtex genutzt werden, dazu werden die bibliographischen Daten in
  einer .bib-Datei gespeichert.
\item Die wichtigsten Literaturdatenbanken für uns sind DBLP und Google Scholar.
\item Google Scholar hat größeren Umfang und erlaubt es, in der Menge der zitierenden
  Artikel zu Suchen. Bei DBLP hat man einen systematischeren Überblick über die Arbeiten eines einzelnen Autoren, sieht besser die verschiedenen Varianten eines Artikels (Konferenz- und Journalversion) und die .bib-Einträge haben höhere Qualität.
\item Sammeln Sie alle gefunden relevanten bibliographischen Daten durchgängig in Ihrem
  Manuskript, nicht erst am Ende der Bearbeitungszeit. Notieren Sie sich eventuell kurz in
  einer eigenen Section zu jedem gefundenen Papier, was darin steht und zitieren Sie es,
  damit es schon in der Literaturliste auftaucht.
\item Die Referenzen sollten im Stil einheitlich sein. Etwa sollten Journalnamen entweder durchgängig ausgeschrieben oder durchgängig abgekürzt sein. Im folgenden befindet sich ein beispielhafter bib-Einträge für einen Konferenzartikel, in einem von uns empfohlenen Stil.
  \begin{quote}
    \texttt{@inproceedings\{MV80,\\
  \quad author    = \{Silvio Micali and
               Vijay V. Vazirani\},\\
  \quad title     = \{An \$\{O(\textbackslash{}sqrt\{|V|\}|E|)\}\$ Algorithm
               for Finding Maximum Matching in General Graphs\},\\
  \quad booktitle = \{Proceedings of the 21st Annual Symposium on Foundations of Computer Science (FOCS$\sim$'80)\},\\
  \quad pages     = \{17--27\},\\
  \quad publisher = \{\{IEEE\} Computer Society\},\\
  year      = 1980,\\
  \}
}
  \end{quote}
\item Man sollte vermeiden, Zitate als Satzteile zu verwenden. Also nicht
  \begin{quote}
    \texttt{In$\sim$\textbackslash{}cite\{MV80\} it was shown that a maximum matching\\ can be computed
    in \$\{O(\textbackslash{}sqrt\{|V|\}|E|)\}\$$\sim$time.}
  \end{quote}
  sondern eher
  \begin{quote}
    \texttt{A maximum matching can be computed\\
    in \$\{O(\textbackslash{}sqrt\{|V|\}|E|)\}\$$\sim$time$\sim$\textbackslash{}cite\{MV80\}.}
  \end{quote}  
\end{itemize}

\subsection{Stilistische Empfehlungen}
\begin{itemize}
\item Wenn Sie einem Objekt eine Bezeichnung gegeben haben, dann verwenden Sie diese Bezeichnung durchgängig und versuchen Sie hier nicht sprachlich zu variieren (das verwirrt den Leser nur).
\item Schreiben Sie "`for example"' anstelle von "`e.g."' und "`,that is,"' anstelle von "`,i.e.,"'.
  
\end{itemize}



\section{Zusammenfassung und Ausblick} 
Hier fasst du nochmals grob deine Ergebnisse zusammen.
Außerdem nennst du einige offene Fragen, die noch interessant sind.











\end{document}